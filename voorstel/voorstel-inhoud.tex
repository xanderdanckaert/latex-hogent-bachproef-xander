%---------- Inleiding ---------------------------------------------------------

\section{Inleiding}
\label{sec:inleiding}

De toepassing van dimensioneringssystemen in magazijnomgevingen is van cruciaal belang voor de efficiënte werking van de logistiek. Het meten van objecten door middel van computer vision-algoritmes biedt een krachtige oplossing om deze processen te verbeteren. Deze bachelorproef richt zich op de optimalisatie van reeds bestaande computer vision-algoritmes, specifiek gericht op het robuuster maken van dimensioneringssystemen, zelfs wanneer de ingevoerde data van mindere kwaliteit is, bijvoorbeeld door slechte verlichting of onscherpe camera's. Daarnaast onderzoekt dit onderzoek ook hoe een systeem automatisch kan worden bijgetraind met nieuwe data die tijdens het gebruik van een applicatie verzameld wordt, zodat het systeem flexibeler en efficiënter wordt in real-world magazijnomgevingen. 

De doelgroep voor dit onderzoek zijn IT-professionals en bedrijven die werken met dimensioneringssystemen in magazijnen en logistieke omgevingen. Het onderzoek richt zich op bedrijven die behoefte hebben aan efficiëntere en robuustere systemen voor het meten van objecten, bijvoorbeeld om optimale opslagplaats te bepalen.

De centrale onderzoeksvraag van dit onderzoek gaat als volgt: 
\begin{quote}
    "Hoe kunnen computer vision-algoritmes geoptimaliseerd en automatisch bijgewerkt worden om dimensioneringssystemen nauwkeuriger en flexibeler te maken?"
\end{quote}

Het doel van dit onderzoek is om een werkend model te ontwikkelen dat toepasbaar is in real-life magazijnomgevingen. Het eindresultaat van deze bachelorproef bestaat uit een robuustere versie van het dimensioneringsalgoritme, een semi-geautomatiseerde trainingsapplicatie en een prototype dat uiteindelijk kan worden geïntegreerd in mobiele applicaties zoals een heftruck.

%---------- Stand van zaken ---------------------------------------------------

\section{Literatuurstudie}
\label{sec:literatuurstudie}

Dimensioneringssystemen die gebruik maken van computer vision-technologie worden steeds vaker toegepast in magazijnen en andere logistieke omgevingen. Deze systemen maken gebruik van camera’s of sensoren om de afmetingen van objecten te schatten. \autocite{getcameras} Echter, een belangrijk probleem is dat de kwaliteit van de inputdata vaak niet voldoet aan de eisen van het algoritme. Onderzoek naar de robuustheid van computer vision-algoritmes toont aan dat slechte verlichting, onscherpte, en andere verstoringen de nauwkeurigheid van dimensioneringssystemen kunnen beïnvloeden \autocite{DeepCorrect}. Daarom is het belangrijk om algoritmes te ontwikkelen die niet alleen kunnen omgaan met hoge kwaliteit beelden, maar ook met mindere kwaliteit data.

Een andere uitdaging is de mogelijkheid om het systeem automatisch bij te trainen met nieuwe data. Dit kan bijvoorbeeld door gebruik te maken van technieken zoals transfer learning of online learning. Deze technieken stellen een model in staat om zich dynamisch aan te passen aan nieuwe omstandigheden zonder dat handmatig gelabelde data nodig is, wat de operationele efficiëntie verhoogt. Verschillende studies hebben aangetoond dat het automatisch bijtrainen van computer vision-modellen potentieel heeft voor toepassingen in dynamische omgevingen, zoals magazijnen, waar de omstandigheden voortdurend veranderen \autocite{AdaptLearn} \autocite{MDPI2}.

Deze technieken zijn al in verschillende domeinen getest, maar de specifieke toepassing voor dimensionering in magazijnen, met de bijkomende eisen voor robuustheid en real-time prestaties, is nog niet uitgebreid onderzocht. Dit maakt het onderwerp van dit onderzoek relevant en vernieuwend.

%---------- Methodologie ------------------------------------------------------

\section{Methodologie}
\label{sec:methodologie}

Om de onderzoeksvraag te beantwoorden, zal dit onderzoek bestaan uit verschillende fasen, beginnend met een grondige analyse van de bestaande computer vision-algoritmes voor dimensionering. In de eerste fase wordt onderzocht welke algoritmes momenteel worden gebruikt en hoe deze geoptimaliseerd kunnen worden voor beelden met een mindere kwaliteit. Er wordt gekeken naar technieken zoals data-augmentatie en ruisonderdrukking om de prestaties van het algoritme te verbeteren.

In de tweede fase wordt een semi-geautomatiseerde trainingsapplicatie ontwikkeld die het mogelijk maakt om het model automatisch bij te trainen met nieuwe data die verzameld wordt tijdens het gebruik van de applicatie. Het systeem wordt ontworpen om in real-time nieuwe gegevens te verwerken en het model dynamisch bij te werken zonder tussenkomst van een gebruiker. 

De laatste fase omvat het testen van het systeem in verschillende magazijnomgevingen, waarbij de applicatie eerst op mobiele telefoons wordt getest en vervolgens geïntegreerd kan worden op een heftruck voor real-time gebruik. Dit zorgt ervoor dat het systeem in verschillende omgevingen en omstandigheden kan functioneren.

De gebruikte onderzoekstechnieken zullen een literatuurstudie, experimenten met computer vision-algoritmes en het ontwikkelen van een prototype voor de trainingsapplicatie omvatten. Voor de ontwikkeling van het systeem worden tools zoals Python, TensorFlow en verschillende hardwarecomponenten (vooral camera’s op mobiele telefoons) ingezet.

De verwachte tijdlijn is als volgt:
\begin{itemize}
    \item Fase 1: Analyse van bestaande algoritmes (1/2 maand)
    \item Fase 2: Ontwikkeling van de trainingsapplicatie en optimalisatie van het algoritme (2 maanden)
    \item Fase 3: Testen van het systeem in een magazijnomgeving (1 maand)
    \item Fase 4: Integratie van het systeem in een stand-alone applicatie en testen op een heftruck (2 maanden)
\end{itemize}

%---------- Verwachte resultaten ----------------------------------------------

\section{Verwacht resultaat, conclusie}
\label{sec:verwachte_resultaten}

De verwachte resultaten van dit onderzoek zijn een geoptimaliseerd en robuust computer vision-algoritme dat in staat is om dimensioneringen nauwkeurig te schatten, zelfs bij mindere kwaliteit beelden. Daarnaast wordt er een semi-geautomatiseerde trainingsapplicatie ontwikkeld die het mogelijk maakt om het model automatisch bij te trainen met nieuwe data. Het prototype van de stand-alone applicatie zal later op een heftruck kunnen worden geïntegreerd.

De meerwaarde van dit onderzoek voor de doelgroep is aanzienlijk. Door het algoritme en de mogelijkheid om het systeem automatisch bij te trainen, kunnen bedrijven hun opslagprocessen efficiënter maken en kosten verlagen. Bovendien biedt het onderzoek een oplossing die eenvoudig geïntegreerd kan worden in bestaande werkprocessen, zoals het gebruik van heftrucks in magazijnen. 

De resultaten zullen niet alleen bijdragen aan de verbetering van dimensioneringssystemen in het algemeen, maar kunnen ook dienen als basis voor verdere ontwikkelingen in automatische bijtraining en adaptieve machine learning-systemen voor andere toepassingen of bedrijfsomgevingen.

\end{document}