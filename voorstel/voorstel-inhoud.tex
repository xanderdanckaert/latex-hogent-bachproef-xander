%---------- Inleiding ---------------------------------------------------------

\section{Inleiding}
\label{sec:inleiding}

De toepassing van dimensioneringssystemen in magazijnomgevingen speelt een sleutelrol in de efficiënte werking van de logistiek \autocite{Cornell2015} \autocite{FreightSnap2019}. Het gebruik van computer vision-algoritmes biedt hierbij een oplossing om objectafmetingen te meten en processen te optimaliseren. Deze bachelorproef richt zich op het evalueren en verbeteren van bestaande computer vision-modellen voor dimensionering, waarbij specifiek wordt gekeken naar hun prestaties onder verschillende omstandigheden, zoals slechte verlichting en onscherpe beelden.

Een belangrijk deel van dit onderzoek is het beoordelen van de robuustheid en accuraatheid van huidige algoritmes. Hierbij wordt onderzocht hoe generiek en noise-resistant bestaande modellen zijn en in hoeverre ze geoptimaliseerd kunnen worden voor magazijnomgevingen. Daarnaast wordt ook gekeken naar mogelijke methodes voor semi-automatische labeling van data, wat van cruciaal belang is voor het trainen van deze algoritmes. Dit omvat het gebruik van bestaande ground truths (verwachte output) en het ontwikkelen van een efficiënte manier om deze te verzamelen.

De doelgroep voor dit onderzoek zijn bedrijven en IT-professionals die werken met dimensioneringssystemen in logistieke omgevingen. Deze bedrijven hebben behoefte aan accuraat werkende systemen om processen efficiënter en kosteneffectiever te maken.

De centrale onderzoeksvraag is:
\begin{quote}
    \emph{"Welke computer vision-algoritmes zijn het meest
    geschikt voor dimensioneringssystemen in
    magazijnomgevingen, en onder welke omstandigheden
    presteren ze optimaal?"}
\end{quote}

Deze vraag wordt ondersteund door deelvragen zoals:
\begin{itemize}
    \item Welke bestaande algoritmes zijn er voor dimensionering, en hoe presteren deze onder verschillende omstandigheden?
    \item Welke datasets zijn geschikt om de prestaties van dimensioneringsalgoritmes te trainen en te valideren?
    \item Hoe kunnen ground truths verkregen worden voor het trainen en evalueren van deze algoritmes?
    \item Welke evaluatietechnieken zijn geschikt om de prestaties van de modellen objectief te meten?
    \item Zijn er bestaande tools of technieken voor (semi-)automatische labeling die toegepast kunnen worden?
    \item Hoe kunnen bestaande dimensioneringsalgoritmes worden aangepast om geïntegreerd te worden in mobiele apparaten zoals een heftruck?
    \item Welke strategieën kunnen worden gebruikt om ruis of andere verstoringen in de inputdata effectief te onderdrukken?

\end{itemize}

Het doel van dit onderzoek is om een gevalideerd en accuraat werkend model te ontwikkelen dat toepasbaar is in real-world magazijnomgevingen. Het resultaat is niet alleen een verbetering van bestaande algoritmes, maar ook een kader waarmee bedrijven deze kunnen evalueren en implementeren.

%---------- Literatuurstudie --------------------------------------------------

\section{Literatuurstudie}
\label{sec:literatuurstudie}

Dimensioneringssystemen op basis van computer vision worden steeds vaker ingezet in logistieke omgevingen vanwege hun potentieel om processen te automatiseren en te optimaliseren \autocite{getcameras2024}. Deze systemen gebruiken sensoren en camera’s om objectafmetingen te meten, maar hun prestaties worden sterk beïnvloed door de kwaliteit van de inputdata. Studies tonen aan dat factoren zoals slechte verlichting, onscherpte en andere vormen van ruis een aanzienlijke impact hebben op de nauwkeurigheid van de algoritmes \autocite{DeepCorrect2024}.

Er zijn verschillende modellen en technieken ontwikkeld om deze uitdagingen te adresseren. Data-augmentatie, ruisonderdrukking en training met diverse datasets zijn bewezen methoden om robuustheid en accuraatheid te verbeteren \autocite{AdaptLearn2024}. Echter, de specifieke toepassing in magazijnomgevingen – waar real-time prestaties en consistentie essentieel zijn – is nog niet uitgebreid onderzocht. Bovendien zijn technieken zoals automatische labeling en het gebruik van bestaande datasets (indien beschikbaar) cruciaal om het labelproces efficiënt en betrouwbaar te maken.

Een bijkomende uitdaging is de evaluatie van modellen. Effectieve evaluatietools en -methodologieën zijn noodzakelijk om objectieve conclusies te trekken over de prestaties van verschillende algoritmes in diverse omstandigheden.

%---------- Methodologie ------------------------------------------------------

\section{Methodologie}
\label{sec:methodologie}

Om de onderzoeksvraag te beantwoorden, wordt dit onderzoek uitgevoerd in drie fasen:
\begin{enumerate}
    \item \textbf{Analyse van bestaande modellen:} In deze fase worden bestaande algoritmes geanalyseerd op hun prestaties bij variabele datakwaliteit. Hierbij wordt gekeken naar robuustheid, noise resistance en generieke toepasbaarheid. De evaluatie maakt hoofdzakelijk gebruik van publieke datasets.
    
    \item \textbf{Data-acquisitie en labeling:} Er wordt onderzocht hoe ground truths effectief verzameld kunnen worden. Dit kan door gebruik te maken van bestaande labels of door semi-geautomatiseerde labelingstechnieken toe te passen. Hierbij wordt rekening gehouden met de uitdagingen van real-world toepassingen. Ook worden bestaande datasets geëvalueerd die geschikt kunnen zijn om de prestaties van dimensioneringsalgoritmes te trainen en te evaluateren.

    \item \textbf{Ontwikkeling en evaluatie:} Op basis van de bevindingen in de eerste twee fasen wordt een verbeterd algoritme ontwikkeld. Het algoritme wordt geëvalueerd met gebruik van specifieke testen, zoals:
    \begin{itemize}
        \item Prestaties bij verschillende licht- en ruisomstandigheden.
        \item Accuraatheid en consistentie bij verschillende objectformaten.
        \item Efficiëntie in real-time toepassingen.
    \end{itemize}
\end{enumerate}

De onderzoeksaanpak omvat een combinatie van literatuurstudie, experimenten met bestaande algoritmes, en het ontwikkelen van een gevalideerd prototype. Tools zoals Python, Torch en OpenCV worden ingezet voor de ontwikkeling en evaluatie.

%---------- Verwachte resultaten ----------------------------------------------

\section{Verwachte resultaten en impact}
\label{sec:verwachte_resultaten}

De verwachte resultaten zijn:
\begin{itemize}
    \item Een gevalideerd en accuraat computer vision-algoritme dat objecten betrouwbaar kan dimensioneren, zelfs bij mindere beeldkwaliteit.
    \item Een effectief kader voor het verkrijgen van ground truths en het evalueren van algoritmes.
    \item Een prototype dat testresultaten valideert in real-world vaak uitdagende magazijnomstandigheden, zonder impact op de veiligheid van processen zoals integratie op heftrucks.
\end{itemize}

De meerwaarde voor bedrijven ligt in een verhoogde accuraatheid en betrouwbaarheid van dimensioneringssystemen, wat leidt tot efficiëntere processen en lagere kosten. Het onderzoek biedt bovendien een fundamentele basis voor verdere ontwikkeling in dit domein.

\vspace{1cm}
\noindent \textbf{Belangrijke kanttekeningen:}
\begin{itemize}
    \item De focus ligt op haalbare en praktische implementaties; theoretische concepten zoals volledig automatische bijtraining worden vermeden.
    \item Het prototype wordt strikt getest op veiligheid en compatibiliteit met logistieke processen.
\end{itemize}
