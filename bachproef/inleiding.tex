%%=============================================================================
%% Inleiding
%%=============================================================================

\chapter{\IfLanguageName{dutch}{Inleiding}{Introduction}}%
\label{ch:inleiding}


De toepassing van dimensioneringssystemen in magazijnomgevingen speelt een sleutelrol in de efficiënte werking van de logistiek \autocite{Cornell2015} \autocite{FreightSnap2019}. Het gebruik van computer vision-algoritmes biedt hierbij een oplossing om objectafmetingen te meten en processen te optimaliseren. Deze bachelorproef richt zich op het evalueren en verbeteren van bestaande computer vision-modellen voor dimensionering, waarbij specifiek wordt gekeken naar hun prestaties onder verschillende omstandigheden, zoals slechte verlichting en onscherpe beelden.

Een belangrijk deel van dit onderzoek is het beoordelen van de robuustheid en accuraatheid van huidige algoritmes. Hierbij wordt onderzocht hoe generiek en noise-resistant bestaande modellen zijn en in hoeverre ze geoptimaliseerd kunnen worden voor magazijnomgevingen. Daarnaast wordt ook gekeken naar mogelijke methodes voor semi-automatische labeling van data, wat van cruciaal belang is voor het trainen van deze algoritmes. Dit omvat het gebruik van bestaande ground truths (verwachte output) en het ontwikkelen van een efficiënte manier om deze te verzamelen.

De doelgroep voor dit onderzoek zijn bedrijven en IT-professionals die werken met dimensioneringssystemen in logistieke omgevingen. Deze bedrijven hebben behoefte aan accuraat werkende systemen om processen efficiënter en kosteneffectiever te maken.



Het doel van dit onderzoek is om een gevalideerd en accuraat werkend model te ontwikkelen dat toepasbaar is in real-world magazijnomgevingen. Het resultaat is niet alleen een verbetering van bestaande algoritmes, maar ook een kader waarmee bedrijven deze kunnen evalueren en implementeren.



\section{\IfLanguageName{dutch}{Probleemstelling}{Problem Statement}}%
\label{sec:probleemstelling}

Uit je probleemstelling moet duidelijk zijn dat je onderzoek een meerwaarde heeft voor een concrete doelgroep. De doelgroep moet goed gedefinieerd en afgelijnd zijn. Doelgroepen als ``bedrijven,'' ``KMO's'', systeembeheerders, enz.~zijn nog te vaag. Als je een lijstje kan maken van de personen/organisaties die een meerwaarde zullen vinden in deze bachelorproef (dit is eigenlijk je steekproefkader), dan is dat een indicatie dat de doelgroep goed gedefinieerd is. Dit kan een enkel bedrijf zijn of zelfs één persoon (je co-promotor/opdrachtgever).

\section{\IfLanguageName{dutch}{Onderzoeksvraag}{Research question}}%
\label{sec:onderzoeksvraag}
De centrale onderzoeksvraag is:
\begin{quote}
    \emph{"Welke computer vision-algoritmes zijn het meest
    geschikt voor dimensioneringssystemen in
    magazijnomgevingen, en onder welke omstandigheden
    presteren ze optimaal?"}
\end{quote}

Deze vraag wordt ondersteund door deelvragen zoals:
\begin{itemize}
    \item Welke bestaande algoritmes zijn er voor dimensionering, en hoe presteren deze onder verschillende omstandigheden?
    \item Welke datasets zijn geschikt om de prestaties van dimensioneringsalgoritmes te trainen en te valideren?
    \item Hoe kunnen ground truths verkregen worden voor het trainen en evalueren van deze algoritmes?
    \item Welke evaluatietechnieken zijn geschikt om de prestaties van de modellen objectief te meten?
    \item Zijn er bestaande tools of technieken voor (semi-)automatische labeling die toegepast kunnen worden?
    \item Hoe kunnen bestaande dimensioneringsalgoritmes worden aangepast om geïntegreerd te worden in mobiele apparaten zoals een heftruck?
    \item Welke strategieën kunnen worden gebruikt om ruis of andere verstoringen in de inputdata effectief te onderdrukken?

\end{itemize}

\section{\IfLanguageName{dutch}{Onderzoeksdoelstelling}{Research objective}}%
\label{sec:onderzoeksdoelstelling}

Het doel van dit onderzoek is om een gevalideerd en accuraat werkend model te ontwikkelen dat toepasbaar is in real-world magazijnomgevingen. Het resultaat is niet alleen een verbetering van bestaande algoritmes, maar ook een kader waarmee bedrijven deze kunnen evalueren en implementeren.

\section{\IfLanguageName{dutch}{Opzet van deze bachelorproef}{Structure of this bachelor thesis}}%
\label{sec:opzet-bachelorproef}

% Het is gebruikelijk aan het einde van de inleiding een overzicht te
% geven van de opbouw van de rest van de tekst. Deze sectie bevat al een aanzet
% die je kan aanvullen/aanpassen in functie van je eigen tekst.

De rest van deze bachelorproef is als volgt opgebouwd:

In Hoofdstuk~\ref{ch:stand-van-zaken} wordt een overzicht gegeven van de stand van zaken binnen het onderzoeksdomein, op basis van een literatuurstudie.

In Hoofdstuk~\ref{ch:methodologie} wordt de methodologie toegelicht en worden de gebruikte onderzoekstechnieken besproken om een antwoord te kunnen formuleren op de onderzoeksvragen.

% TODO: Vul hier aan voor je eigen hoofstukken, één of twee zinnen per hoofdstuk

In Hoofdstuk~\ref{ch:conclusie}, tenslotte, wordt de conclusie gegeven en een antwoord geformuleerd op de onderzoeksvragen. Daarbij wordt ook een aanzet gegeven voor toekomstig onderzoek binnen dit domein.