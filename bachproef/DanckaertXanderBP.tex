%===============================================================================
% LaTeX sjabloon voor de bachelorproef toegepaste informatica aan HOGENT
% Meer info op https://github.com/HoGentTIN/latex-hogent-report
%===============================================================================

\documentclass[dutch,dit,thesis]{hogentreport}

% TODO:
% - If necessary, replace the option `dit`' with your own department!
%   Valid entries are dbo, dbt, dgz, dit, dlo, dog, dsa, soa
% - If you write your thesis in English (remark: only possible after getting
%   explicit approval!), remove the option "dutch," or replace with "english".

\usepackage{lipsum} % For blind text, can be removed after adding actual content

%% Pictures to include in the text can be put in the graphics/ folder
\graphicspath{{../graphics/}}

%% For source code highlighting, requires pygments to be installed
%% Compile with the -shell-escape flag!
%% \usepackage[chapter]{minted}
%% If you compile with the make_thesis.{bat,sh} script, use the following
%% import instead:
\usepackage[chapter,outputdir=../output]{minted}
\usemintedstyle{solarized-light}

%% Formatting for minted environments.
\setminted{%
    autogobble,
    frame=lines,
    breaklines,
    linenos,
    tabsize=4
}

%% Ensure the list of listings is in the table of contents
\renewcommand\listoflistingscaption{%
    \IfLanguageName{dutch}{Lijst van codefragmenten}{List of listings}
}
\renewcommand\listingscaption{%
    \IfLanguageName{dutch}{Codefragment}{Listing}
}
\renewcommand*\listoflistings{%
    \cleardoublepage\phantomsection\addcontentsline{toc}{chapter}{\listoflistingscaption}%
    \listof{listing}{\listoflistingscaption}%
}

% Other packages not already included can be imported here

%%---------- Document metadata -------------------------------------------------
% TODO: Replace this with your own information
\author{Xander Danckaert}
\supervisor{Dr. Gilles Vandewiele}
\cosupervisor{Dhr. Simon De Gheselle}
\title[Optionele ondertitel]%
    {Optimalisatie van Computer Vision en Automatische Bijtraining voor Dimensioneringssystemen}
\academicyear{\advance\year by -1 \the\year--\advance\year by 1 \the\year}
\examperiod{1}
\degreesought{\IfLanguageName{dutch}{Professionele bachelor in de toegepaste informatica}{Bachelor of applied computer science}}
\partialthesis{false} %% To display 'in partial fulfilment'
%\institution{Internshipcompany BVBA.}

%% Add global exceptions to the hyphenation here
\hyphenation{back-slash}

%% The bibliography (style and settings are  found in hogentthesis.cls)
\addbibresource{bachproef.bib}            %% Bibliography file
\addbibresource{../voorstel/voorstel.bib} %% Bibliography research proposal
\defbibheading{bibempty}{}

%% Prevent empty pages for right-handed chapter starts in twoside mode
\renewcommand{\cleardoublepage}{\clearpage}

\renewcommand{\arraystretch}{1.2}

%% Content starts here.
\begin{document}

%---------- Front matter -------------------------------------------------------

\frontmatter

\hypersetup{pageanchor=false} %% Disable page numbering references
%% Render a Dutch outer title page if the main language is English
\IfLanguageName{english}{%
    %% If necessary, information can be changed here
    \degreesought{Professionele Bachelor toegepaste informatica}%
    \begin{otherlanguage}{dutch}%
       \maketitle%
    \end{otherlanguage}%
}{}

%% Generates title page content
\maketitle
\hypersetup{pageanchor=true}

\input{voorwoord}
\input{samenvatting}

%---------- Inhoud, lijst figuren, ... -----------------------------------------

\tableofcontents

% In a list of figures, the complete caption will be included. To prevent this,
% ALWAYS add a short description in the caption!
%
%  \caption[short description]{elaborate description}
%
% If you do, only the short description will be used in the list of figures

\listoffigures

% If you included tables and/or source code listings, uncomment the appropriate
% lines.
\listoftables

\listoflistings

% Als je een lijst van afkortingen of termen wil toevoegen, dan hoort die
% hier thuis. Gebruik bijvoorbeeld de ``glossaries'' package.
% https://www.overleaf.com/learn/latex/Glossaries

%---------- Kern ---------------------------------------------------------------

\mainmatter{}

% De eerste hoofdstukken van een bachelorproef zijn meestal een inleiding op
% het onderwerp, literatuurstudie en verantwoording methodologie.
% Aarzel niet om een meer beschrijvende titel aan deze hoofdstukken te geven of
% om bijvoorbeeld de inleiding en/of stand van zaken over meerdere hoofdstukken
% te verspreiden!

%%=============================================================================
%% Inleiding
%%=============================================================================

\chapter{\IfLanguageName{dutch}{Inleiding}{Introduction}}%
\label{ch:inleiding}


De toepassing van dimensioneringssystemen in magazijnomgevingen speelt een sleutelrol in de efficiënte werking van de logistiek \autocite{Cornell2015} \autocite{FreightSnap2019}. Het gebruik van computer vision-algoritmes biedt hierbij een oplossing om objectafmetingen te meten en processen te optimaliseren. Deze bachelorproef richt zich op het evalueren en verbeteren van bestaande computer vision-modellen voor dimensionering, waarbij specifiek wordt gekeken naar hun prestaties onder verschillende omstandigheden, zoals slechte verlichting en onscherpe beelden.

Een belangrijk deel van dit onderzoek is het beoordelen van de robuustheid en accuraatheid van huidige algoritmes. Hierbij wordt onderzocht hoe generiek en noise-resistant bestaande modellen zijn en in hoeverre ze geoptimaliseerd kunnen worden voor magazijnomgevingen. Daarnaast wordt ook gekeken naar mogelijke methodes voor semi-automatische labeling van data, wat van cruciaal belang is voor het trainen van deze algoritmes. Dit omvat het gebruik van bestaande ground truths (verwachte output) en het ontwikkelen van een efficiënte manier om deze te verzamelen.

De doelgroep voor dit onderzoek zijn bedrijven en IT-professionals die werken met dimensioneringssystemen in logistieke omgevingen. Deze bedrijven hebben behoefte aan accuraat werkende systemen om processen efficiënter en kosteneffectiever te maken.



Het doel van dit onderzoek is om een gevalideerd en accuraat werkend model te ontwikkelen dat toepasbaar is in real-world magazijnomgevingen. Het resultaat is niet alleen een verbetering van bestaande algoritmes, maar ook een kader waarmee bedrijven deze kunnen evalueren en implementeren.



\section{\IfLanguageName{dutch}{Probleemstelling}{Problem Statement}}%
\label{sec:probleemstelling}

Uit je probleemstelling moet duidelijk zijn dat je onderzoek een meerwaarde heeft voor een concrete doelgroep. De doelgroep moet goed gedefinieerd en afgelijnd zijn. Doelgroepen als ``bedrijven,'' ``KMO's'', systeembeheerders, enz.~zijn nog te vaag. Als je een lijstje kan maken van de personen/organisaties die een meerwaarde zullen vinden in deze bachelorproef (dit is eigenlijk je steekproefkader), dan is dat een indicatie dat de doelgroep goed gedefinieerd is. Dit kan een enkel bedrijf zijn of zelfs één persoon (je co-promotor/opdrachtgever).

\section{\IfLanguageName{dutch}{Onderzoeksvraag}{Research question}}%
\label{sec:onderzoeksvraag}
De centrale onderzoeksvraag is:
\begin{quote}
    \emph{"Welke computer vision-algoritmes zijn het meest
    geschikt voor dimensioneringssystemen in
    magazijnomgevingen, en onder welke omstandigheden
    presteren ze optimaal?"}
\end{quote}

Deze vraag wordt ondersteund door deelvragen zoals:
\begin{itemize}
    \item Welke bestaande algoritmes zijn er voor dimensionering, en hoe presteren deze onder verschillende omstandigheden?
    \item Welke datasets zijn geschikt om de prestaties van dimensioneringsalgoritmes te trainen en te valideren?
    \item Hoe kunnen ground truths verkregen worden voor het trainen en evalueren van deze algoritmes?
    \item Welke evaluatietechnieken zijn geschikt om de prestaties van de modellen objectief te meten?
    \item Zijn er bestaande tools of technieken voor (semi-)automatische labeling die toegepast kunnen worden?
    \item Hoe kunnen bestaande dimensioneringsalgoritmes worden aangepast om geïntegreerd te worden in mobiele apparaten zoals een heftruck?
    \item Welke strategieën kunnen worden gebruikt om ruis of andere verstoringen in de inputdata effectief te onderdrukken?

\end{itemize}

\section{\IfLanguageName{dutch}{Onderzoeksdoelstelling}{Research objective}}%
\label{sec:onderzoeksdoelstelling}

Het doel van dit onderzoek is om een gevalideerd en accuraat werkend model te ontwikkelen dat toepasbaar is in real-world magazijnomgevingen. Het resultaat is niet alleen een verbetering van bestaande algoritmes, maar ook een kader waarmee bedrijven deze kunnen evalueren en implementeren.

\section{\IfLanguageName{dutch}{Opzet van deze bachelorproef}{Structure of this bachelor thesis}}%
\label{sec:opzet-bachelorproef}

% Het is gebruikelijk aan het einde van de inleiding een overzicht te
% geven van de opbouw van de rest van de tekst. Deze sectie bevat al een aanzet
% die je kan aanvullen/aanpassen in functie van je eigen tekst.

De rest van deze bachelorproef is als volgt opgebouwd:

In Hoofdstuk~\ref{ch:stand-van-zaken} wordt een overzicht gegeven van de stand van zaken binnen het onderzoeksdomein, op basis van een literatuurstudie.

In Hoofdstuk~\ref{ch:methodologie} wordt de methodologie toegelicht en worden de gebruikte onderzoekstechnieken besproken om een antwoord te kunnen formuleren op de onderzoeksvragen.

% TODO: Vul hier aan voor je eigen hoofstukken, één of twee zinnen per hoofdstuk

In Hoofdstuk~\ref{ch:conclusie}, tenslotte, wordt de conclusie gegeven en een antwoord geformuleerd op de onderzoeksvragen. Daarbij wordt ook een aanzet gegeven voor toekomstig onderzoek binnen dit domein.
\input{standvanzaken}
\input{methodologie}

% Voeg hier je eigen hoofdstukken toe die de ``corpus'' van je bachelorproef
% vormen. De structuur en titels hangen af van je eigen onderzoek. Je kan bv.
% elke fase in je onderzoek in een apart hoofdstuk bespreken.

%\input{...}
%\input{...}
%...

\input{conclusie}

%---------- Bijlagen -----------------------------------------------------------

\appendix

\chapter{Onderzoeksvoorstel}

Het onderwerp van deze bachelorproef is gebaseerd op een onderzoeksvoorstel dat vooraf werd beoordeeld door de promotor. Dat voorstel is opgenomen in deze bijlage.

%% TODO: 
%\section*{Samenvatting}

% Kopieer en plak hier de samenvatting (abstract) van je onderzoeksvoorstel.

% Verwijzing naar het bestand met de inhoud van het onderzoeksvoorstel
%---------- Inleiding ---------------------------------------------------------

\section{Inleiding}
\label{sec:inleiding}

De toepassing van dimensioneringssystemen in magazijnomgevingen is van cruciaal belang voor de efficiënte werking van de logistiek. Het meten van objecten door middel van computer vision-algoritmes biedt een krachtige oplossing om deze processen te verbeteren. Deze bachelorproef richt zich op de optimalisatie van reeds bestaande computer vision-algoritmes, specifiek gericht op het robuuster maken van dimensioneringssystemen, zelfs wanneer de ingevoerde data van mindere kwaliteit is, bijvoorbeeld door slechte verlichting of onscherpe camera's. Daarnaast onderzoekt dit onderzoek ook hoe een systeem automatisch kan worden bijgetraind met nieuwe data die tijdens het gebruik van een applicatie verzameld wordt, zodat het systeem flexibeler en efficiënter wordt in real-world magazijnomgevingen. 

De doelgroep voor dit onderzoek zijn IT-professionals en bedrijven die werken met dimensioneringssystemen in magazijnen en logistieke omgevingen. Het onderzoek richt zich op bedrijven die behoefte hebben aan efficiëntere en robuustere systemen voor het meten van objecten, bijvoorbeeld om optimale opslagplaats te bepalen.

De centrale onderzoeksvraag van dit onderzoek gaat als volgt: 
\begin{quote}
    "Hoe kunnen computer vision-algoritmes geoptimaliseerd en automatisch bijgewerkt worden om dimensioneringssystemen nauwkeuriger en flexibeler te maken?"
\end{quote}

Het doel van dit onderzoek is om een werkend model te ontwikkelen dat toepasbaar is in real-life magazijnomgevingen. Het eindresultaat van deze bachelorproef bestaat uit een robuustere versie van het dimensioneringsalgoritme, een semi-geautomatiseerde trainingsapplicatie en een prototype dat uiteindelijk kan worden geïntegreerd in mobiele applicaties zoals een heftruck.

%---------- Stand van zaken ---------------------------------------------------

\section{Literatuurstudie}
\label{sec:literatuurstudie}

Dimensioneringssystemen die gebruik maken van computer vision-technologie worden steeds vaker toegepast in magazijnen en andere logistieke omgevingen. Deze systemen maken gebruik van camera’s of sensoren om de afmetingen van objecten te schatten. \autocite{getcameras2024} Echter, een belangrijk probleem is dat de kwaliteit van de inputdata vaak niet voldoet aan de eisen van het algoritme. Onderzoek naar de robuustheid van computer vision-algoritmes toont aan dat slechte verlichting, onscherpte, en andere verstoringen de nauwkeurigheid van dimensioneringssystemen kunnen beïnvloeden \autocite{DeepCorrect2024}. Daarom is het belangrijk om algoritmes te ontwikkelen die niet alleen kunnen omgaan met hoge kwaliteit beelden, maar ook met mindere kwaliteit data.

Een andere uitdaging is de mogelijkheid om het systeem automatisch bij te trainen met nieuwe data. Dit kan bijvoorbeeld door gebruik te maken van technieken zoals transfer learning of online learning. Deze technieken stellen een model in staat om zich dynamisch aan te passen aan nieuwe omstandigheden zonder dat handmatig gelabelde data nodig is, wat de operationele efficiëntie verhoogt. Verschillende studies hebben aangetoond dat het automatisch bijtrainen van computer vision-modellen potentieel heeft voor toepassingen in dynamische omgevingen, zoals magazijnen, waar de omstandigheden voortdurend veranderen \autocite{AdaptLearn2024} \autocite{MDPI2024}.

Deze technieken zijn al in verschillende domeinen getest, maar de specifieke toepassing voor dimensionering in magazijnen, met de bijkomende eisen voor robuustheid en real-time prestaties, is nog niet uitgebreid onderzocht. Dit maakt het onderwerp van dit onderzoek relevant en vernieuwend.

%---------- Methodologie ------------------------------------------------------

\section{Methodologie}
\label{sec:methodologie}

Om de onderzoeksvraag te beantwoorden, zal dit onderzoek bestaan uit verschillende fasen, beginnend met een grondige analyse van de bestaande computer vision-algoritmes voor dimensionering. In de eerste fase wordt onderzocht welke algoritmes momenteel worden gebruikt en hoe deze geoptimaliseerd kunnen worden voor beelden met een mindere kwaliteit. Er wordt gekeken naar technieken zoals data-augmentatie en ruisonderdrukking om de prestaties van het algoritme te verbeteren.

In de tweede fase wordt een semi-geautomatiseerde trainingsapplicatie ontwikkeld die het mogelijk maakt om het model automatisch bij te trainen met nieuwe data die verzameld wordt tijdens het gebruik van de applicatie. Het systeem wordt ontworpen om in real-time nieuwe gegevens te verwerken en het model dynamisch bij te werken zonder tussenkomst van een gebruiker. 

De laatste fase omvat het testen van het systeem in verschillende magazijnomgevingen, waarbij de applicatie eerst op mobiele telefoons wordt getest en vervolgens geïntegreerd kan worden op een heftruck voor real-time gebruik. Dit zorgt ervoor dat het systeem in verschillende omgevingen en omstandigheden kan functioneren.

De gebruikte onderzoekstechnieken zullen een literatuurstudie, experimenten met computer vision-algoritmes en het ontwikkelen van een prototype voor de trainingsapplicatie omvatten. Voor de ontwikkeling van het systeem worden tools zoals Python, TensorFlow en verschillende hardwarecomponenten (vooral camera’s op mobiele telefoons) ingezet.

De verwachte tijdlijn is als volgt:
\begin{itemize}
    \item Fase 1: Analyse van bestaande algoritmes (1/2 maand)
    \item Fase 2: Ontwikkeling van de trainingsapplicatie en optimalisatie van het algoritme (2 maanden)
    \item Fase 3: Testen van het systeem in een magazijnomgeving (1 maand)
    \item Fase 4: Integratie van het systeem in een stand-alone applicatie en testen op een heftruck (2 maanden)
\end{itemize}

%---------- Verwachte resultaten ----------------------------------------------

\section{Verwacht resultaat, conclusie}
\label{sec:verwachte_resultaten}

De verwachte resultaten van dit onderzoek zijn een geoptimaliseerd en robuust computer vision-algoritme dat in staat is om dimensioneringen nauwkeurig te schatten, zelfs bij mindere kwaliteit beelden. Daarnaast wordt er een semi-geautomatiseerde trainingsapplicatie ontwikkeld die het mogelijk maakt om het model automatisch bij te trainen met nieuwe data. Het prototype van de stand-alone applicatie zal later op een heftruck kunnen worden geïntegreerd.

De meerwaarde van dit onderzoek voor de doelgroep is aanzienlijk. Door het algoritme en de mogelijkheid om het systeem automatisch bij te trainen, kunnen bedrijven hun opslagprocessen efficiënter maken en kosten verlagen. Bovendien biedt het onderzoek een oplossing die eenvoudig geïntegreerd kan worden in bestaande werkprocessen, zoals het gebruik van heftrucks in magazijnen. 

De resultaten zullen niet alleen bijdragen aan de verbetering van dimensioneringssystemen in het algemeen, maar kunnen ook dienen als basis voor verdere ontwikkelingen in automatische bijtraining en adaptieve machine learning-systemen voor andere toepassingen of bedrijfsomgevingen.



%%---------- Andere bijlagen --------------------------------------------------
% TODO: Voeg hier eventuele andere bijlagen toe. Bv. als je deze BP voor de
% tweede keer indient, een overzicht van de verbeteringen t.o.v. het origineel.
%\input{...}

%%---------- Backmatter, referentielijst ---------------------------------------

\backmatter{}

\setlength\bibitemsep{2pt} %% Add Some space between the bibliograpy entries
\printbibliography[heading=bibintoc]

\end{document}
